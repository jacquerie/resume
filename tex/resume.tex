\documentclass[12pt,a4paper,sans]{moderncv}

\moderncvstyle{classic}
\moderncvcolor{red}
\moderncvicons{awesome}

\usepackage{fullpage}
\usepackage[utf8]{inputenc}

\name{Jacopo}{Notarstefano}
\phone[fixed]{+33~603635095}
\email{jacopo.notarstefano@gmail.com}
\homepage{jacquerie.github.io}

\begin{document}

  \makecvtitle

  %------------------------
  \section{Work Experience}
  %------------------------

  \cventry{2016 -- 2018}{Junior Fellow}{CERN}{Geneva}{Switzerland}{
    I continued contributing to INSPIRE, taking additional responsibilities
    such as student selection and supervision. Among other projects,  I was
    involved in
    \begin{itemize}
      \item a general rewrite of the application in order to follow common
        Flask patterns for large applications and ensure future maintainability;
      \item a Docker-based development environment and testing infrastructure
        in order to ease the frequent onboarding of new students and achieve
        reproducible builds;
      \item a pytest-based test suite that spanned from style enforcement to UI
        tests in order to reduce the rate of defects and prevent regressions.
    \end{itemize}
  }

  \vspace{0.25cm}

  \cventry{2015 -- 2016}{Technical Student}{CERN}{Geneva}{Switzerland}{
    I started contributing to INSPIRE, the leading information platform for
    the High Energy Physics community. In particular I developed
    \begin{itemize}
      \item \texttt{invenio-matcher}, a module that leverages Elasticsearch to
        detect if a newly ingested record is a duplicate of a record already
        present in the system;
      \item \texttt{invenio-grobid}, a module that wraps the GROBID Machine
        Learning library to extract metadata from PDFs.
    \end{itemize}
  }

  \vspace{0.25cm}

  \cventry{2014 -- 2015}{Contract Developer}{University of Pisa}{Pisa}{Italy}{
    I built CAPS, a web application that the Department of Mathematics of the
    University of Pisa is still using to collect and process study plans. It
    features a REST backend API written using the CakePHP framework and a
    jQuery-based frontend that interacts with it.
  }

  %-------------------------
  \section{Other Experience}
  %-------------------------

  \cvitem{\textbf{Python}}{    
    I used the standard scientific stack (Jupyter, NumPy, SciPy...) and a
    Scrapy crawler to show that a filibustering technique employed by an
    Italian Senator could be neutered by a simple Machine Learning algorithm.
    This work earned me national newspaper coverage and was eventually
    integrated in the procedures of the Italian Senate.
  }

  %------------------
  \section{Education}
  %------------------

  \cventry{}{Master's Degree in Computer Science}{University of Pisa}{}{in progress}{}

  \cventry{}{Bachelor's Degree in Mathematics}{University of Pisa}{}{107/110}{}

\end{document}
