\documentclass[12pt,a4paper,sans]{moderncv}

\moderncvstyle{classic}
\moderncvcolor{red}
\moderncvicons{awesome}

\usepackage{fullpage}
\usepackage[utf8x]{inputenc}
\usepackage{lmodern}
\usepackage{textcomp}

\name{Jacopo}{Notarstefano}
\phone[fixed]{+33~603635095}
\email{jacopo.notarstefano@gmail.com}
\homepage{jacquerie.github.io}

\begin{document}

  \makecvtitle

  %------------------------
  \section{Work Experience}
  %------------------------

  \cventry{\footnotesize{Feb~16~-~Jun~18}}{Junior Fellow}{CERN}{Geneva}{Switzerland}{
    I contributed to INSPIRE, the leading information platform for the High
    Energy Physics community, taking additional responsibilities like student
    selection and supervision. Among other projects, I was involved in:
    \begin{itemize}
      \item a general rewrite of the application, in order to follow common
        Flask patterns for large applications and ensure future maintainability;
      \item a Docker-based development environment and testing infrastructure,
        in order to ease the frequent onboarding of new students and achieve
        reproducible builds;
      \item a test suite based on Pytest spanning from style enforcement to UI
        tests, in order to reduce the rate of defects and prevent regressions;
      \item a module to migrate the data from the legacy MARCXML format to JSON;
      \item an Elasticsearch-based module  to count the number of citations of
        a record;
      \item a module based on Scikit-Learn to detect when two author signatures
        in distinct records actually refer to the same author.
    \end{itemize}
  }

  \vspace{0.25cm}

  \cventry{\footnotesize{Feb~15~-~Nov~15}}{Technical Student}{CERN}{Geneva}{Switzerland}{
    I started contributing to INSPIRE. In particular I developed:
    \begin{itemize}
      \item an Elasticsearch-based module to detect when a newly ingested
        record is actually a duplicate of a record already present in the
        system;
      \item a module based on the GROBID Machine Learning library to
        automatically extract record metadata from harvested PDFs.
    \end{itemize}
  }

  \vspace{0.25cm}

  \cventry{\footnotesize{Apr~14~-~Jan~15}}{Contract Developer}{University of Pisa}{Pisa}{Italy}{
    I built CAPS, a web application that the Department of Mathematics of the
    University of Pisa is still using to collect and process study plans. It
    features a REST backend API written using the CakePHP framework and a
    jQuery-based frontend that interacts with it.
  }

  %------------------
  \section{Education}
  %------------------

  \cventry{\footnotesize{May~12~-~Mar~14}}{Master's Degree in Computer Science}{University of Pisa}{}{abandoned}{}

  \cventry{\footnotesize{Sep~06~-~Apr~12}}{Bachelor's Degree in Mathematics}{University of Pisa}{}{107/110}{}

  %---------------
  \section{Awards}
  %---------------

  \cventry{\footnotesize{Sep 06}}{Istituto Nazionale di Alta Matematica}{}{}{accepted a €12000 scholarship}{}

  \cventry{\footnotesize{Sep 06}}{Scuola Galileiana di Studi Superiori}{}{}{declined a full scholarship offer}{}

  \cventry{\footnotesize{May 06}}{Kangourou della Matematica}{}{}{won first prize in the national finals}{}

  \cventry{\footnotesize{May 06}}{Olimpiadi della Matematica}{}{}{won a gold medal in the national finals}{}

\end{document}
